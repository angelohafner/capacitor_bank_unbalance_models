\documentclass[a4paper]{article}

% ======================================================
% FONT
% ======================================================
\usepackage{helvet}
\renewcommand{\familydefault}{\sfdefault}

% ======================================================
% LANGUAGE AND MATH
% ======================================================
\usepackage[portuguese]{babel}
\usepackage{amsmath}
\usepackage{amssymb}

% ======================================================
% GRAPHICS AND TABLES
% ======================================================
\usepackage{graphicx}
\usepackage{array}
\usepackage{booktabs}
\usepackage{xcolor}

% ======================================================
% PAGE LAYOUT
% ======================================================
\usepackage{geometry}
\usepackage{fancyhdr}
\usepackage{lastpage}
\usepackage{indentfirst}
\usepackage{setspace}
\usepackage{enumitem}

\geometry{
	left=20mm,
	right=20mm,
	top=40mm,
	bottom=30mm,
	headsep=20mm
}

% ======================================================
% DATE FORMAT
% ======================================================
\usepackage[datesep=/,style=ddmmyyyy]{datetime2}

% ======================================================
% LINKS
% ======================================================
\usepackage{hyperref}
\usepackage{xurl}
\usepackage{url}

% ======================================================
% BIBLIOGRAPHY
% ======================================================
\usepackage[style=ieee]{biblatex}
\addbibresource{bibliografia.bib}

% ======================================================
% SECTION STYLE
% ======================================================
\usepackage{titlesec}
\titleformat{\section}
{\normalfont\large\bfseries}{\thesection}{1em}{}

% ======================================================
% HEADER / FOOTER
% ======================================================
\pagestyle{fancy}
\fancyhf{}
\renewcommand{\headrulewidth}{0pt}
\renewcommand{\footrulewidth}{0.4pt}

\fancyhead[C]{
	\begin{tabular}{|m{3.5cm}|m{9cm}|m{3.5cm}|}
		\hline
		\begin{minipage}[c][2cm][c]{3.5cm}
			\centering
			\includegraphics[width=2.98cm,height=1.25cm]{./figs/logo.png}
		\end{minipage}
		&
		\begin{minipage}[c][2cm][c]{9cm}
			\centering
			\hyphenpenalty=10000
			\vspace*{\fill}
			\begin{spacing}{1.5}
				{\large \textbf{Proteção de Banco de\\ Capacitores em Ponte H}}
			\end{spacing}
			\vspace*{\fill}
		\end{minipage}
		&
		\begin{minipage}[c][2cm][c]{3.5cm}
			\raggedleft
			Emissão: \DTMtoday\\
			Folha: \thepage/\pageref{LastPage}
		\end{minipage}
		\\
		\hline
	\end{tabular}
}

\fancyfoot[L]{\href{http://www.dax.energy}{www.dax.energy}}
\fancyfoot[C]{\href{mailto:comercial@dax.energy}{comercial@dax.energy}}
\fancyfoot[R]{+55 41 99281-3744}

% ======================================================
\begin{document}
	\setstretch{1.25}
	
	% ======================================================
	\section{Contexto}
	% ======================================================
	
	Bancos de capacitores conectados em ponte H com fusíveis externos são empregados em sistemas de média e alta tensão quando se busca elevada confiabilidade operacional associada à facilidade de manutenção e substituição de unidades capacitivas.
	
	Devido às tolerâncias construtivas, variações de tensão do sistema e diferenças naturais entre unidades capacitivas, pequenos desequilíbrios elétricos são esperados durante a operação normal. Nessas condições, a corrente diferencial da ponte permanece reduzida, caracterizando regime equilibrado.
	
	A ocorrência de falhas em unidades capacitivas ou em seus fusíveis externos altera o balanço elétrico entre os ramos da ponte H, produzindo corrente diferencial mensurável. Essa variação constitui o princípio fundamental de detecção de anormalidades nesse tipo de banco.
	
	\begin{figure}[htbp]
		\centering
		\includegraphics[width=0.6\linewidth]{./figs/Figure32-h_bridge_external_fuses}
		\caption{Configuração típica de banco de capacitores em ponte H com fusíveis externos.}
	\end{figure}
	
	Na condição ideal de operação, os ramos opostos da ponte apresentam impedâncias equivalentes, resultando em diferença de potencial praticamente nula entre os pontos médios e, consequentemente, corrente diferencial próxima de zero.
	
	% ======================================================
	\section{Banco em Estudo}
	% ======================================================
	
	\begin{figure}[htbp]
		\centering
		\includegraphics[height=8cm,width=16cm]{./figs/bank_diagram_matplotlib}
		\caption{Diagrama elétrico do banco de capacitores em ponte H analisado.}
		\label{fig:h_bridge_bank}
	\end{figure}
	
	O banco analisado utiliza configuração em ponte H composta por unidades capacitivas protegidas por fusíveis externos, conforme ilustrado na Fig.~\ref{fig:h_bridge_bank}.
	
	Essa topologia apresenta as seguintes características:
	
	\begin{itemize}
		\item elevada sensibilidade à perda de unidades capacitivas;
		\item facilidade de manutenção e substituição de equipamentos;
		\item redução do impacto térmico sobre as unidades capacitivas;
		\item elevada robustez operacional.
	\end{itemize}
	
	Os fusíveis externos são instalados em série com cada unidade capacitiva e têm como função isolar rapidamente o capacitor defeituoso em caso de falha, evitando a propagação de danos ao restante do banco.
	
	\begin{table}[htbp]
		\centering
		\renewcommand{\arraystretch}{1.25}
		\caption{Dados nominais do banco de capacitores.}
		\centering
\begin{tabular}{lr}
\hline
\textbf{Variavel} & \textbf{Valor} \\
\hline
Aterrado (0) / Isolado (1) & 1  \\
Elementos internos em paralelo no grupo & 16  \\
Frequencia (Hz) & 60  \\
Grupos de elementos em serie em uma unidade & 3  \\
Potencia Nominal (VAr) & 19 M \\
Potencia de Trabalho (VAr) & 15 M \\
Tensao Nominal (V) & 40 k \\
Tensao de Trabalho (V) & 35.5 k \\
Unidades Paralelas Ramo Esquerda & 5  \\
Unidades Paralelas String Afetada & 2  \\
Unidades Paralelas Total & 9  \\
Unidades Series Fase-Neutro & 3  \\
\hline
\end{tabular}

	\end{table}
	
	% ======================================================
	\section{Princípio de Detecção de Desequilíbrio}
	% ======================================================
	
	Na configuração em ponte H com fusíveis externos, a proteção baseia-se na medição da corrente diferencial entre os ramos da ponte.
	
	Em condição equilibrada:
	
	\[
	I_{\text{diff}} \approx 0
	\]
	
	A atuação de um fusível externo ou a perda de uma unidade capacitiva provoca alteração da impedância do ramo correspondente, resultando em corrente diferencial proporcional ao grau de desequilíbrio.
	
	A medição é normalmente realizada por:
	
	\begin{itemize}
		\item Transformador de Corrente diferencial instalado no elo central da ponte;
		\item Funções dedicadas de proteção de desequilíbrio em relés digitais.
	\end{itemize}
	
	Esse método permite detectar tanto falhas internas quanto operações de fusíveis externos, garantindo rápida identificação de condições anormais.
	
	% ======================================================
	\section{Níveis de Supervisão}
	% ======================================================
	
	O comportamento da corrente diferencial é classificado em três níveis operacionais:
	
	\subsection*{Alarme}
	
	Indica desequilíbrio inicial associado à perda limitada de unidades capacitivas. Recomenda-se inspeção programada e acompanhamento operacional.
	
	\subsection*{Preocupação}
	
	Caracteriza aumento progressivo do desequilíbrio, podendo resultar em sobretensões nas unidades remanescentes.
	
	\subsection*{Desligamento}
	
	Quando o limite crítico é atingido, o banco deve ser desligado automaticamente para evitar sobrecargas dielétricas e falhas em cascata.
	
	\begin{table}[htbp]
		\small
		\centering
		\caption{Níveis típicos de alarme e desligamento para corrente diferencial da ponte H.}
		\begin{tabular}{rrrrrrrrrrrr}
\toprule
\textit{\textbf{f}} & \textbf{\textit{C\textsubscript{p}} [pu]} & \textbf{\textit{C\textsubscript{p}} [$\mu$F]} & \textbf{\textit{C\textsubscript{hn}} [pu]} & \textbf{\textit{C\textsubscript{hn}} [$\mu$F]} & \textbf{\textit{V\textsubscript{hn}} [pu]} & \textbf{\textit{V\textsubscript{hn}} [V]} & \textbf{\textit{I\textsubscript{h}} [pu]} & \textbf{\textit{I\textsubscript{h}} [A]} & \textbf{\textit{V\textsubscript{cu}} [pu]} & \textbf{\textit{V\textsubscript{cu}} [kV]} & \textbf{\textit{V\textsubscript{cu2}} [pu2]} \\
\midrule
\textcolor{red}{0} & \textcolor{red}{3,00} & \textcolor{red}{94,72} & \textcolor{red}{5,00} & \textcolor{red}{157,86} & \textcolor{red}{0,60} & \textcolor{red}{12297,56} & \textcolor{red}{0,00} & \textcolor{red}{0,00} & \textcolor{red}{1,00} & \textcolor{red}{4,10} & \textcolor{red}{0,89} \\
1 & 2,96 & 93,33 & 4,88 & 154,03 & 0,61 & 12417,97 & 0,01 & 2,80 & 1,11 & 4,54 & 0,98 \\
2 & 2,90 & 91,62 & 4,73 & 149,44 & 0,61 & 12565,63 & 0,03 & 6,27 & 1,24 & 5,08 & 1,10 \\
3 & 2,83 & 89,48 & 4,56 & 143,83 & 0,62 & 12750,93 & 0,04 & 10,69 & 1,41 & 5,77 & 1,25 \\
4 & 2,75 & 86,71 & 4,33 & 136,81 & 0,63 & 12990,38 & 0,07 & 16,50 & 1,63 & 6,68 & 1,45 \\
5 & 2,63 & 83,00 & 4,05 & 127,79 & 0,65 & 13311,80 & 0,10 & 24,48 & 1,94 & 7,93 & 1,72 \\
6 & 2,46 & 77,75 & 3,67 & 115,76 & 0,67 & 13765,93 & 0,15 & 36,14 & 2,38 & 9,76 & 2,11 \\
7 & 2,21 & 69,78 & 3,13 & 98,93 & 0,71 & 14456,38 & 0,22 & 54,77 & 3,09 & 12,68 & 2,74 \\
\bottomrule
\end{tabular}
		\label{tab:hbridge_levels}
	\end{table}
	
	A filosofia de proteção estabelece que nenhuma célula capacitiva deve operar acima de \(110\%\) da tensão nominal.
	
	\begin{itemize}
		\item valores em azul correspondem aos níveis de alarme;
		\item valores em vermelho correspondem aos níveis de desligamento automático.
	\end{itemize}
	
	Recomenda-se transformador de corrente diferencial com corrente primária {{ctprimaryrated}} A e relação {{ctprimaryrated}}:{{secondarycurrenttc}}.
	
	% ======================================================
	\section{Fator de Segurança}
	% ======================================================
	
	Assim como em outras configurações de bancos de capacitores, é prática comum especificar capacitores com tensão nominal superior à tensão de operação do sistema.
	
	Principais benefícios:
	
	\begin{enumerate}
		\item redução do estresse dielétrico;
		\item aumento da vida útil das unidades capacitivas;
		\item maior tolerância a transitórios e sobretensões temporárias.
	\end{enumerate}
	
	A potência reativa fornecida permanece definida por
	
	\[
	Q = V^{2} C \omega
	\]
	
	dependendo exclusivamente da tensão efetiva aplicada e da capacitância instalada.
	
	% ======================================================
	\printbibliography
	% ======================================================
	
	\vspace{2cm}
	
	\noindent
	\begin{minipage}[t]{0.3\textwidth}
		\centering
		\vspace{4cm}
		\rule{5cm}{0.4pt}\\
		\textbf{Angelo A. Hafner}\\
		\small Engenheiro Eletricista\\
		Doutor em Eletromagnetismo\\
		CONFEA: 2.500.821.919\\
		CREA/SC: 045.776-5
	\end{minipage}
	\hfill
	\begin{minipage}[t]{0.3\textwidth}
		\centering
		\vspace{4cm}
		\rule{5cm}{0.4pt}\\
		\textbf{Daniel H. Pires}\\
		\small Engenheiro Eletricista\\
		Especialista em Sistemas de Potência\\
		CONFEA: 7.718.547.498\\
		CREA/PR: 179.137/D
	\end{minipage}
	\hfill
	\begin{minipage}[t]{0.3\textwidth}
		\centering
		\vspace{4cm}
		\rule{5cm}{0.4pt}\\
		\textbf{Fabricio R. Frangiotti}\\
		\small Engenheiro Eletricista\\
		Especialista em Capacitores de Potência\\
		CREA/SP: 5061.769.065/D
	\end{minipage}
	
\end{document}