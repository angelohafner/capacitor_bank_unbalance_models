\documentclass[a4paper]{article}

% ======================================================
% FONT AND LANGUAGE
% ======================================================
\usepackage{helvet}
\renewcommand{\familydefault}{\sfdefault}

\usepackage[portuguese]{babel}

% ======================================================
% MATH AND SYMBOLS
% ======================================================
\usepackage{amsmath}
\usepackage{amssymb}

% ======================================================
% GRAPHICS AND TABLES
% ======================================================
\usepackage{graphicx}
\usepackage{array}
\usepackage{booktabs}
\usepackage{xcolor}

% ======================================================
% PAGE STYLE
% ======================================================
\usepackage{geometry}
\usepackage{fancyhdr}
\usepackage{lastpage}
\usepackage{indentfirst}
\usepackage{setspace}
\usepackage{enumitem}

\geometry{
	left=20mm,
	right=20mm,
	top=40mm,
	bottom=30mm,
	headsep=20mm
}

% ======================================================
% DATE FORMAT
% ======================================================
\usepackage[datesep=/,style=ddmmyyyy]{datetime2}

% ======================================================
% LINKS
% ======================================================
\usepackage{hyperref}
\usepackage{xurl}
\usepackage{url}

% ======================================================
% BIBLIOGRAPHY
% ======================================================
\usepackage[style=ieee]{biblatex}
\addbibresource{./bib/bibliografia.bib}

% ======================================================
% SECTION FORMAT
% ======================================================
\usepackage{titlesec}

\titleformat{\section}
{\normalfont\large\bfseries}
{\thesection}{1em}{}

% ======================================================
% HEADER AND FOOTER
% ======================================================
\pagestyle{fancy}
\fancyhf{}

\renewcommand{\headrulewidth}{0pt}
\renewcommand{\footrulewidth}{0.4pt}

\fancyhead[C]{
	\begin{tabular}{|m{3.5cm}|m{9cm}|m{3.5cm}|}
		\hline
		\begin{minipage}[c][2cm][c]{3.5cm}
			\centering
			\includegraphics[width=2.98cm,height=1.25cm]{./figs/logo.png}
		\end{minipage}
		&
		\begin{minipage}[c][2cm][c]{9cm}
			\centering
			\hyphenpenalty=10000
			\vspace*{\fill}
			\begin{spacing}{1.5}
				{\large \textbf{Proteção de Banco de Capacitores}}
			\end{spacing}
			\vspace*{\fill}
		\end{minipage}
		&
		\begin{minipage}[c][2cm][c]{3.5cm}
			\raggedleft
			Emissão: \DTMtoday\\
			Folha: \thepage/\pageref{LastPage}
		\end{minipage}
		\\
		\hline
	\end{tabular}
}

\fancyfoot[L]{\href{http://www.dax.energy}{www.dax.energy}}
\fancyfoot[C]{\href{mailto:comercial@dax.energy}{comercial@dax.energy}}
\fancyfoot[R]{+55 41 99281-3744}

% ======================================================
% DOCUMENT
% ======================================================
\begin{document}
	
	\setstretch{1.25}
	
	% ======================================================
	\section{Contexto}
	% ======================================================
	
	Em sistemas elétricos de potência, especialmente em bancos de capacitores de média tensão, pequenos desequilíbrios operacionais são considerados inerentes ao funcionamento normal. Tais variações podem decorrer de tolerâncias construtivas dos capacitores, diferenças de tensão entre fases ou pequenas assimetrias de carga. Quando mantidos dentro de limites reduzidos, esses efeitos não comprometem a segurança nem o desempenho do sistema.
	
	Os bancos de capacitores são projetados para operar de forma segura mesmo na presença desses desvios. Entretanto, o monitoramento contínuo permanece necessário para garantir que os níveis permaneçam dentro das faixas admissíveis. Sistemas adequados de supervisão permitem a detecção antecipada de anomalias e aumentam a confiabilidade operacional.
	
	\begin{figure}[htbp]
		\centering
		\includegraphics[width=0.9\linewidth]{./figs/Figure34-yy_internal_fuses}
		\caption{Ilustração de um banco em dupla estrela assimétrica \cite{ieeec3799}.}
		\label{fig:figure-34-illustration-of-an-uneven-double-wye-connected-bank}
	\end{figure}
	
	Em configurações em estrela isolada, o neutro não é diretamente aterrado. Em regime normal, a tensão neutro-terra \(V_{OG}\) e a corrente de desequilíbrio \(I_n\) apresentam valores muito baixos, idealmente próximos de zero. Falhas internas ou a perda parcial de elementos capacitivos provocam aumento dessas grandezas, indicando necessidade de avaliação técnica.
	
	O monitoramento pode ser realizado por meio de transformador de potencial, em estrela simples isolada, ou por transformador de corrente, em dupla estrela isolada.
	
	% ======================================================
	\section{Banco em Tela}
	% ======================================================
	
	\begin{figure}[htbp]
		\centering
		\includegraphics[height=9cm]{./figs/bank_diagram_matplotlib}
		\caption{Banco em análise.}
		\label{fig:banco-em-tela}
	\end{figure}
	
	O banco apresentado na Fig.~\ref{fig:banco-em-tela} utiliza unidades capacitivas com fusíveis internos, solução amplamente empregada em sistemas de média tensão para compensação de potência reativa, melhoria do fator de potência e redução de perdas elétricas.
	
	Os fusíveis internos permitem a desconexão automática do elemento defeituoso, evitando propagação de danos e mantendo a operação do conjunto até a intervenção programada.
	
	\begin{table}[htbp]
		\centering
		\renewcommand{\arraystretch}{1.25}
		\caption{Dados de entrada do banco de capacitores.}
		\centering
\begin{tabular}{lr}
\hline
\textbf{Variavel} & \textbf{Valor} \\
\hline
Aterrado (0) / Isolado (1) & 1  \\
Elementos internos em paralelo no grupo & 16  \\
Frequencia (Hz) & 60  \\
Grupos de elementos em serie em uma unidade & 3  \\
Potencia Nominal (VAr) & 19 M \\
Potencia de Trabalho (VAr) & 15 M \\
Tensao Nominal (V) & 40 k \\
Tensao de Trabalho (V) & 35.5 k \\
Unidades Paralelas Ramo Esquerda & 5  \\
Unidades Paralelas String Afetada & 2  \\
Unidades Paralelas Total & 9  \\
Unidades Series Fase-Neutro & 3  \\
\hline
\end{tabular}

		\label{tab:tabela7_real}
	\end{table}
	
	% ======================================================
	\section{Critérios de Alarme e Ação}
	% ======================================================
	
	O aumento progressivo do desequilíbrio operacional define diferentes níveis de atuação:
	
	\begin{itemize}
		\item \textbf{Alarme}: indica condição potencialmente anormal e requer análise preventiva.
		\item \textbf{Preocupação}: pode afetar desempenho e segurança, exigindo inspeção detalhada.
		\item \textbf{Ação}: condição crítica que requer desligamento automático do banco.
	\end{itemize}
	
	% ======================================================
	\section{TC ou TP de Neutro}
	% ======================================================
	
	A Tabela~\ref{tab:df_subset} apresenta cenários operacionais considerando perda de elementos capacitivos, conforme diretrizes do \href{https://ieeexplore.ieee.org/document/6466331}{IEEE Std 18 \cite{ieee18}}.
	
	\begin{table}[htbp]
		\setlength{\tabcolsep}{5pt}
		\scriptsize
		\centering
		\caption{Tabela de desbalanço — níveis de alarme e desligamento automático.}
		\begin{tabular}{rrrrrrrrrrrrrr}
\toprule
\textit{\textbf{f}} & \textbf{\textit{C\textsubscript{p}} [pu]} & \textbf{\textit{C\textsubscript{p}} [$\mu$F]} & \textbf{\textit{C\textsubscript{u}} [pu]} & \textbf{\textit{C\textsubscript{u}} [$\mu$F]} & \textbf{\textit{V\textsubscript{ng}} [pu]} & \textbf{\textit{V\textsubscript{ng}} [V]} & \textbf{\textit{I\textsubscript{n}} [pu]} & \textbf{\textit{I\textsubscript{n}} [A]} & \textbf{\textit{I\textsubscript{g}} [pu]} & \textbf{\textit{I\textsubscript{g}} [A]} & \textbf{\textit{V\textsubscript{cu}} [pu]} & \textbf{\textit{V\textsubscript{cu}} [kV]} & \textbf{\textit{V\textsubscript{cu2}} [pu2]} \\
\midrule
0 & 1,00 & 31,57 & 1,00 & 11,48 & 0,00 & 0,00 & 0,00 & 0,00 & 0,00 & 0,00 & 1,00 & 5,12 & 0,89 \\
1 & 1,00 & 31,55 & 0,97 & 11,19 & 0,00 & 3,91 & 0,00 & 0,06 & 0,00 & 0,00 & 1,01 & 5,16 & 0,89 \\
2 & 1,00 & 31,53 & 0,95 & 10,88 & 0,00 & 8,28 & 0,00 & 0,13 & 0,00 & 0,00 & 1,01 & 5,19 & 0,90 \\
3 & 1,00 & 31,51 & 0,92 & 10,52 & 0,00 & 13,22 & 0,00 & 0,21 & 0,00 & 0,00 & 1,02 & 5,24 & 0,91 \\
4 & 1,00 & 31,49 & 0,88 & 10,13 & 0,00 & 18,84 & 0,00 & 0,31 & 0,00 & 0,00 & 1,03 & 5,28 & 0,92 \\
5 & 1,00 & 31,46 & 0,84 & 9,69 & 0,00 & 25,27 & 0,00 & 0,41 & 0,00 & 0,00 & 1,04 & 5,34 & 0,92 \\
\textcolor{blue}{6} & \textcolor{blue}{1,00} & \textcolor{blue}{31,42} & \textcolor{blue}{0,80} & \textcolor{blue}{9,18} & \textcolor{blue}{0,00} & \textcolor{blue}{32,73} & \textcolor{blue}{0,00} & \textcolor{blue}{0,53} & \textcolor{blue}{0,00} & \textcolor{blue}{0,00} & \textcolor{blue}{1,05} & \textcolor{blue}{5,40} & \textcolor{blue}{0,94} \\
7 & 0,99 & 31,38 & 0,75 & 8,61 & 0,00 & 41,48 & 0,00 & 0,67 & 0,00 & 0,00 & 1,07 & 5,48 & 0,95 \\
\textcolor{red}{8} & \textcolor{red}{0,99} & \textcolor{red}{31,33} & \textcolor{red}{0,69} & \textcolor{red}{7,95} & \textcolor{red}{0,00} & \textcolor{red}{51,90} & \textcolor{red}{0,00} & \textcolor{red}{0,84} & \textcolor{red}{0,00} & \textcolor{red}{0,00} & \textcolor{red}{1,09} & \textcolor{red}{5,57} & \textcolor{red}{0,96} \\
9 & 0,99 & 31,28 & 0,62 & 7,18 & 0,00 & 64,46 & 0,00 & 1,05 & 0,00 & 0,00 & 1,11 & 5,67 & 0,98 \\
10 & 0,99 & 31,20 & 0,55 & 6,26 & 0,00 & 79,93 & 0,01 & 1,30 & 0,00 & 0,00 & 1,13 & 5,80 & 1,01 \\
11 & 0,99 & 31,11 & 0,45 & 5,17 & 0,00 & 99,49 & 0,01 & 1,61 & 0,00 & 0,00 & 1,17 & 5,97 & 1,03 \\
12 & 0,98 & 31,00 & 0,33 & 3,83 & 0,01 & 124,98 & 0,01 & 2,03 & 0,00 & 0,00 & 1,21 & 6,19 & 1,07 \\
13 & 0,98 & 30,84 & 0,19 & 2,15 & 0,01 & 159,54 & 0,01 & 2,59 & 0,00 & 0,00 & 1,26 & 6,48 & 1,12 \\
\bottomrule
\end{tabular}
		\label{tab:df_subset}
	\end{table}
	
	Cada célula possui {{N}} elementos em paralelo e {{Su}} em série, garantindo adequada distribuição elétrica e maior confiabilidade operacional.
	
	A filosofia de proteção estabelece que a tensão em qualquer célula não deve exceder \(10\%\) da tensão nominal. Valores em \textcolor{blue}{azul} indicam alarme e valores em \textcolor{red}{vermelho} indicam desligamento.
	
	Recomenda-se TC de neutro com corrente nominal {{ctprimaryrated}} A e relação {{ctprimaryrated}}:{{secondarycurrenttc}}.
	
	Alternativamente, pode-se utilizar TP com tensão nominal {{ptprimaryrated}} V e relação {{ptprimaryrated}}:{{secondaryvoltagetp}} para monitoramento da tensão de deslocamento do neutro.
	
	% ======================================================
	\section{Fator de Segurança}
	% ======================================================
	
	A utilização de capacitores com tensão nominal superior à tensão de operação é prática comum devido a:
	
	\begin{enumerate}
		\item maior margem dielétrica;
		\item aumento da vida útil;
		\item maior flexibilidade de aplicação.
	\end{enumerate}
	
	A potência reativa fornecida é dada por
	
	\[
	Q = V^{2} \cdot C \cdot \omega
	\]
	
	dependendo da tensão aplicada e não da tensão nominal do capacitor.
	
	% ======================================================
	\printbibliography
	% ======================================================
	
	\vspace{2cm}
	
	\noindent
	\begin{minipage}[t]{0.3\textwidth}
		\centering
		\vspace{4cm}
		\rule{5cm}{0.4pt}\\
		\textbf{Angelo A. Hafner}\\
		\small
		Engenheiro Eletricista\\
		Doutor em Eletromagnetismo\\
		CONFEA: 2.500.821.919\\
		CREA/SC: 045.776-5
	\end{minipage}
	\hfill
	\begin{minipage}[t]{0.3\textwidth}
		\centering
		\vspace{4cm}
		\rule{5cm}{0.4pt}\\
		\textbf{Daniel H. Pires}\\
		\small
		Engenheiro Eletricista\\
		Especialista em Sistemas de Potência\\
		CONFEA: 7.718.547.498\\
		CREA/PR: 179.137/D
	\end{minipage}
	\hfill
	\begin{minipage}[t]{0.3\textwidth}
		\centering
		\vspace{4cm}
		\rule{5cm}{0.4pt}\\
		\textbf{Fabricio R. Frangiotti}\\
		\small
		Engenheiro Eletricista\\
		Especialista em Capacitores de Potência\\
		CREA/SP: 5061.769.065/D
	\end{minipage}
	
\end{document}