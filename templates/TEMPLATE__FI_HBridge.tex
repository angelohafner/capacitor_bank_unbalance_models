\documentclass[a4paper]{article}

% ======================================================
% FONT
% ======================================================
\usepackage{helvet}
\renewcommand{\familydefault}{\sfdefault}

% ======================================================
% LANGUAGE AND MATH
% ======================================================
\usepackage[portuguese]{babel}
\usepackage{amsmath}
\usepackage{amssymb}

% ======================================================
% GRAPHICS AND TABLES
% ======================================================
\usepackage{graphicx}
\usepackage{array}
\usepackage{booktabs}
\usepackage{xcolor}

% ======================================================
% PAGE LAYOUT
% ======================================================
\usepackage{geometry}
\usepackage{fancyhdr}
\usepackage{lastpage}
\usepackage{indentfirst}
\usepackage{setspace}
\usepackage{enumitem}

\geometry{
	left=20mm,
	right=20mm,
	top=40mm,
	bottom=30mm,
	headsep=20mm
}

% ======================================================
% DATE FORMAT
% ======================================================
\usepackage[datesep=/,style=ddmmyyyy]{datetime2}

% ======================================================
% LINKS
% ======================================================
\usepackage{hyperref}
\usepackage{xurl}
\usepackage{url}

% ======================================================
% BIBLIOGRAPHY
% ======================================================
\usepackage[style=ieee]{biblatex}
\addbibresource{bibliografia.bib}

% ======================================================
% SECTION STYLE
% ======================================================
\usepackage{titlesec}
\titleformat{\section}
{\normalfont\large\bfseries}{\thesection}{1em}{}

% ======================================================
% HEADER / FOOTER
% ======================================================
\pagestyle{fancy}
\fancyhf{}
\renewcommand{\headrulewidth}{0pt}
\renewcommand{\footrulewidth}{0.4pt}

\fancyhead[C]{
	\begin{tabular}{|m{3.5cm}|m{9cm}|m{3.5cm}|}
		\hline
		\begin{minipage}[c][2cm][c]{3.5cm}
			\centering
			\includegraphics[width=2.98cm,height=1.25cm]{./figs/logo.png}
		\end{minipage}
		&
		\begin{minipage}[c][2cm][c]{9cm}
			\centering
			\hyphenpenalty=10000
			\vspace*{\fill}
			\begin{spacing}{1.5}
				{\large \textbf{Proteção de Banco de\\ Capacitores em Ponte H}}
			\end{spacing}
			\vspace*{\fill}
		\end{minipage}
		&
		\begin{minipage}[c][2cm][c]{3.5cm}
			\raggedleft
			Emissão: \DTMtoday\\
			Folha: \thepage/\pageref{LastPage}
		\end{minipage}
		\\
		\hline
	\end{tabular}
}

\fancyfoot[L]{\href{http://www.dax.energy}{www.dax.energy}}
\fancyfoot[C]{\href{mailto:comercial@dax.energy}{comercial@dax.energy}}
\fancyfoot[R]{+55 41 99281-3744}

% ======================================================
\begin{document}
	\setstretch{1.25}
	
	% ======================================================
	\section{Contexto}
	% ======================================================
	
	Bancos de capacitores conectados em ponte H são amplamente utilizados em sistemas de média e alta tensão quando se deseja elevada sensibilidade na detecção de falhas internas e maior estabilidade frente a desequilíbrios naturais do sistema elétrico.
	
	Pequenos desvios de capacitância entre unidades são inevitáveis devido às tolerâncias construtivas, envelhecimento dielétrico e variações operacionais. Em regime normal, tais desequilíbrios produzem correntes diferenciais muito reduzidas na ponte, não comprometendo o desempenho do banco.
	
	Entretanto, a perda parcial de elementos capacitivos ou falhas internas provoca alteração significativa do equilíbrio elétrico da ponte H, resultando no surgimento de corrente diferencial mensurável. Essa característica torna a configuração em ponte H particularmente adequada para esquemas de proteção sensíveis.
	
	\begin{figure}[htbp]
		\centering
		\includegraphics[width=0.9\linewidth]{./figs/Figure36-h_bridge_internal_fuses}
		\caption{Configuração típica de banco de capacitores em ponte H com fusíveis internos.}
	\end{figure}
	
	A filosofia de operação baseia-se no princípio de equilíbrio elétrico entre os ramos da ponte. Em condição ideal, a diferença de potencial entre os pontos médios é próxima de zero, resultando em corrente diferencial nula.
	
	% ======================================================
	\section{Banco em Estudo}
	% ======================================================
	
	\begin{figure}[htbp]
		\centering
		\includegraphics[height=8cm, width=16cm]{./figs/bank_diagram_matplotlib}
		\caption{Diagrama elétrico do banco de capacitores em ponte H analisado.}
		\label{fig:h_bridge_bank}
	\end{figure}
	
	O banco analisado é constituído por unidades capacitivas com fusíveis internos conectadas em configuração ponte H, conforme ilustrado na Fig.~\ref{fig:h_bridge_bank}.
	
	Essa topologia apresenta as seguintes características:
	
	\begin{itemize}
		\item elevada sensibilidade para detecção de falhas internas;
		\item operação estável mesmo com pequenas assimetrias naturais;
		\item continuidade operacional após falha individual de elementos;
		\item eliminação da necessidade de neutro aterrado.
	\end{itemize}
	
	Os fusíveis internos promovem a desconexão automática do elemento defeituoso, evitando propagação da falha e preservando a integridade do banco.
	
	\begin{table}[htbp]
		\centering
		\renewcommand{\arraystretch}{1.25}
		\caption{Dados nominais do banco de capacitores.}
		\centering
\begin{tabular}{lr}
\hline
\textbf{Variavel} & \textbf{Valor} \\
\hline
Aterrado (0) / Isolado (1) & 1  \\
Elementos internos em paralelo no grupo & 16  \\
Frequencia (Hz) & 60  \\
Grupos de elementos em serie em uma unidade & 3  \\
Potencia Nominal (VAr) & 19 M \\
Potencia de Trabalho (VAr) & 15 M \\
Tensao Nominal (V) & 40 k \\
Tensao de Trabalho (V) & 35.5 k \\
Unidades Paralelas Ramo Esquerda & 5  \\
Unidades Paralelas String Afetada & 2  \\
Unidades Paralelas Total & 9  \\
Unidades Series Fase-Neutro & 3  \\
\hline
\end{tabular}

	\end{table}
	
	% ======================================================
	\section{Princípio de Detecção de Desequilíbrio}
	% ======================================================
	
	Na configuração em ponte H, a proteção baseia-se na medição da corrente diferencial entre os pontos médios da ponte.
	
	Em operação equilibrada:
	\(
	I_{\text{diff}} \approx 0.
	\)
	A perda de elementos capacitivos altera a impedância de um dos ramos, produzindo corrente diferencial proporcional ao nível de desequilíbrio.
	
	A medição pode ser realizada por meio de:
	
	\begin{itemize}
		\item Transformador de Corrente diferencial instalado no elo da ponte;
		\item Sistema de proteção dedicado à função de desequilíbrio de capacitor.
	\end{itemize}
	
	Essa técnica permite detectar falhas incipientes antes que ocorram sobretensões perigosas nas unidades restantes.
	
	% ======================================================
	\section{Níveis de Supervisão}
	% ======================================================
	
	O comportamento da corrente diferencial é normalmente dividido em três níveis operacionais:
	
	\subsection*{Alarme}
	
	Indica surgimento de desequilíbrio inicial. Recomenda-se inspeção programada e acompanhamento da tendência operacional.
	
	\subsection*{Preocupação}
	
	Caracteriza degradação progressiva das unidades capacitivas, exigindo análise técnica detalhada.
	
	\subsection*{Desligamento}
	
	Quando a corrente diferencial ultrapassa o limite admissível, o banco deve ser desligado automaticamente para evitar sobrecarga dielétrica e falhas em cascata.
	
	\begin{table}[htbp]
		\small
		\centering
		\caption{Níveis típicos de alarme e desligamento para corrente diferencial da ponte H.}
		\begin{tabular}{rrrrrrrrrrrr}
\toprule
\textit{\textbf{f}} & \textbf{\textit{C\textsubscript{p}} [pu]} & \textbf{\textit{C\textsubscript{p}} [$\mu$F]} & \textbf{\textit{C\textsubscript{u}} [pu]} & \textbf{\textit{C\textsubscript{u}} [$\mu$F]} & \textbf{\textit{C\textsubscript{hn}} [pu]} & \textbf{\textit{C\textsubscript{hn}} [$\mu$F]} & \textbf{\textit{I\textsubscript{h}} [pu]} & \textbf{\textit{I\textsubscript{h}} [A]} & \textbf{\textit{V\textsubscript{cu}} [pu]} & \textbf{\textit{V\textsubscript{cu}} [kV]} & \textbf{\textit{V\textsubscript{cu2}} [pu2]} \\
\midrule
0 & 1,29 & 40,59 & 1,00 & 24,56 & 3,00 & 94,72 & 0,00 & 0,00 & 1,00 & 2,93 & 0,89 \\
1 & 1,29 & 40,58 & 0,98 & 24,02 & 3,00 & 94,64 & 0,00 & 0,09 & 1,01 & 2,95 & 0,89 \\
2 & 1,28 & 40,56 & 0,95 & 23,44 & 2,99 & 94,55 & 0,00 & 0,19 & 1,02 & 2,98 & 0,90 \\
3 & 1,28 & 40,54 & 0,93 & 22,80 & 2,99 & 94,46 & 0,00 & 0,29 & 1,03 & 3,01 & 0,91 \\
4 & 1,28 & 40,53 & 0,90 & 22,10 & 2,99 & 94,35 & 0,00 & 0,42 & 1,04 & 3,04 & 0,92 \\
5 & 1,28 & 40,50 & 0,87 & 21,32 & 2,98 & 94,23 & 0,00 & 0,55 & 1,05 & 3,07 & 0,93 \\
\textcolor{blue}{6} & \textcolor{blue}{1,28} & \textcolor{blue}{40,48} & \textcolor{blue}{0,83} & \textcolor{blue}{20,46} & \textcolor{blue}{2,98} & \textcolor{blue}{94,10} & \textcolor{blue}{0,00} & \textcolor{blue}{0,71} & \textcolor{blue}{1,06} & \textcolor{blue}{3,11} & \textcolor{blue}{0,94} \\
7 & 1,28 & 40,45 & 0,79 & 19,50 & 2,98 & 93,94 & 0,00 & 0,89 & 1,08 & 3,16 & 0,96 \\
\textcolor{red}{8} & \textcolor{red}{1,28} & \textcolor{red}{40,42} & \textcolor{red}{0,75} & \textcolor{red}{18,42} & \textcolor{red}{2,97} & \textcolor{red}{93,76} & \textcolor{red}{0,00} & \textcolor{red}{1,10} & \textcolor{red}{1,10} & \textcolor{red}{3,22} & \textcolor{red}{0,98} \\
9 & 1,28 & 40,38 & 0,70 & 17,19 & 2,96 & 93,55 & 0,01 & 1,35 & 1,12 & 3,28 & 0,99 \\
10 & 1,28 & 40,33 & 0,64 & 15,79 & 2,96 & 93,29 & 0,01 & 1,65 & 1,15 & 3,36 & 1,02 \\
11 & 1,28 & 40,27 & 0,58 & 14,17 & 2,95 & 92,99 & 0,01 & 2,00 & 1,18 & 3,45 & 1,05 \\
12 & 1,27 & 40,20 & 0,50 & 12,28 & 2,93 & 92,61 & 0,01 & 2,45 & 1,22 & 3,57 & 1,08 \\
13 & 1,27 & 40,11 & 0,41 & 10,05 & 2,92 & 92,13 & 0,01 & 3,01 & 1,27 & 3,72 & 1,13 \\
14 & 1,27 & 39,99 & 0,30 & 7,37 & 2,90 & 91,51 & 0,02 & 3,76 & 1,34 & 3,91 & 1,19 \\
15 & 1,26 & 39,83 & 0,17 & 4,09 & 2,87 & 90,67 & 0,02 & 4,78 & 1,43 & 4,18 & 1,27 \\
\bottomrule
\end{tabular}
		\label{tab:hbridge_levels}
	\end{table}
	
	A filosofia de proteção estabelece que nenhuma célula capacitiva deve operar acima de \(110\%\) da tensão nominal.
	
	\begin{itemize}
		\item valores em azul representam níveis de alarme;
		\item valores em vermelho representam níveis de desligamento automático.
	\end{itemize}
	
	Sugere-se transformador de corrente diferencial com corrente primária {{ctprimaryrated}} A e relação {{ctprimaryrated}}:{{secondarycurrenttc}}.
	
	% ======================================================
	\section{Fator de Segurança}
	% ======================================================
	
	Assim como em outras topologias de bancos de capacitores, é prática comum especificar capacitores com tensão nominal superior à tensão de operação do sistema.
	
	Principais benefícios:
	
	\begin{enumerate}
		\item redução do estresse dielétrico;
		\item aumento da vida útil dos capacitores;
		\item maior robustez frente a transitórios e sobretensões temporárias.
	\end{enumerate}
	
	A potência reativa fornecida permanece definida por	
	\(
	Q = V^{2} C \omega
	\)
	dependendo da tensão efetivamente aplicada e da capacitância instalada.
	
	% ======================================================
	\printbibliography
	% ======================================================
	
	\vspace{2cm}
	
	\noindent
	\begin{minipage}[t]{0.3\textwidth}
		\centering
		\vspace{4cm}
		\rule{5cm}{0.4pt}\\
		\textbf{Angelo A. Hafner}\\
		\small Engenheiro Eletricista\\
		Doutor em Eletromagnetismo\\
		CONFEA: 2.500.821.919\\
		CREA/SC: 045.776-5
	\end{minipage}
	\hfill
	\begin{minipage}[t]{0.3\textwidth}
		\centering
		\vspace{4cm}
		\rule{5cm}{0.4pt}\\
		\textbf{Daniel H. Pires}\\
		\small Engenheiro Eletricista\\
		Especialista em Sistemas de Potência\\
		CONFEA: 7.718.547.498\\
		CREA/PR: 179.137/D
	\end{minipage}
	\hfill
	\begin{minipage}[t]{0.3\textwidth}
		\centering
		\vspace{4cm}
		\rule{5cm}{0.4pt}\\
		\textbf{Fabricio R. Frangiotti}\\
		\small Engenheiro Eletricista\\
		Especialista em Capacitores de Potência\\
		CREA/SP: 5061.769.065/D
	\end{minipage}
	
\end{document}